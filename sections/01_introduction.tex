\section{Introduction}
Combinatorial search problems are fundamental to computer science and artificial intelligence, with applications spanning logistics, scheduling, bioinformatics, and puzzle-solving. The Futoshiki puzzle, a Japanese constraint satisfaction problem, serves as an excellent benchmark for evaluating algorithmic approaches to these challenges. As an NP-Complete problem \cite{Sen2024Futoshiki}, Futoshiki requires filling an N×N grid with numbers from 1 to N while satisfying two constraint sets: the Latin Square property (each number appears exactly once per row and column) and inequality constraints between adjacent cells.

While simple backtracking algorithms guarantee a solution, their exponential time complexity renders them impractical for large grids. A more sophisticated approach, proposed by Şen and Diner \cite{Sen2024Futoshiki}, transforms the puzzle into a list coloring problem. This paradigm introduces a pre-coloring phase that propagates constraints to reduce possible values for each cell, significantly pruning the search tree before the recursive search begins.

However, even with these optimizations, solving large or exceptionally difficult puzzles demands substantial computational resources. High-Performance Computing (HPC) offers a path to overcoming these limitations through parallelization across multiple processors. This paper presents a comprehensive parallel computing framework for the Futoshiki puzzle that explores three distinct parallelization paradigms:

\begin{enumerate}
    \item \textbf{Shared-Memory Parallelism (OpenMP):} Exploiting multi-core architectures through task-based parallelism within a single node.
    \item \textbf{Distributed-Memory Parallelism (MPI):} Scaling across multiple nodes using a dynamic master-worker model.
    \item \textbf{Hybrid Parallelism (MPI+OpenMP):} Combining both paradigms to maximize resource utilization in modern HPC clusters.
\end{enumerate}

Our key contributions include:
\begin{itemize}
    \item An efficient sequential solver leveraging pre-coloring optimization as a performance baseline.
    \item A unified work generation framework that dynamically partitions the search space based on available parallelism.
    \item Three parallel implementations with configurable task generation factors for workload tuning.
    \item Comprehensive performance analysis including speedup, efficiency, and scalability metrics across all paradigms.
\end{itemize}