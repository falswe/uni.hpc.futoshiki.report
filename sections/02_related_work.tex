\section{Related Work}
\label{sec:related_work}
As our work builds upon and extends the solutions presented in the Futoshiki paper, in order for the reader to have a better understanding on the terminologies and techniques used, in this section we present all of the necessary relevant informations, also proposing how our solution is going to impact and build upon the state of the art (or concepts) presented in each subsection. 

We start in \Cref{subsec:csp} by giving an explanation of the Constraint Satisfaction and Propagation. In \Cref{subsec:backtrack} then move towards a brief recap of how backtrack can be leveraged to solve this problem. We continue in explaining how MPI and OMP are used in 'Irregular Problems' in \Cref{subsec:mpi_omp_irregular_problems}. We move to an high level view of the state of puzzle solving algorithms in \Cref{subsec:puzzle_solving_algorithms}. We then wrap up by presenting how load balancing is used in parallel computing in \Cref{subsec:load_balancing_in_parallel_computing}.

\subsection{Constraint Satisfaction and Propagation}
\label{subsec:csp}
The foundation of our approach lies in constraint propagation techniques: After Adamu-Fika showed in \cite{sudoku_csp} that it is possible to transform Sudoku leveraging CSP, Norvig's work on Sudoku solving \cite{NorvigSudoku} demonstrated the power of constraint propagation combined with search, achieving several fold speedups over naive backtracking. Our pre-coloring phase extends these ideas specifically for Futoshiki's inequality constraints.
\sd{here we say reduction of search space in polynomial time blablabla}

The transformation of Futoshiki into a list coloring problem, as proposed by Şen and Diner \cite{Sen2024Futoshiki}, provides the theoretical foundation for our implementation. Their work showed that viewing the puzzle through the lens of graph coloring enables more sophisticated pruning strategies compared to naive backtracking strategies. 

The main concept in which this solution is proposed is stated in \textit{Theorem 1} of such paper, which affirms that for every instance of the Futoshiki problem \textbf{F\_n(T,S)} -- where an instance is defined by denoting with \textbf{S} the Set of inequality constraints, \textbf{T} the set of assigned cells, and \textbf{n} the grid size -- exists a List Precoloring instance \textbf{$\mathcal{L}$\_G(c\_w,L)} -- where \textbf{L} is the List of valid assignments and \textbf{c\_w} the map to assign each vertex a color -- such that k=n. They also show that this approach effectively reduces the total execution time w.r.t. the naif brute force backtracking approach.

Our work therefore starts from here, by implementing their sequential algorithm proposed in this paper, in order to construct a base reference for our solutions. We then follow up by implementing parallel solving strategies such as MPI, OMP, and an hybrid solution.


One might ask himself "Why do we need to parallalize this problem if there is already a linear solution?". The response is quite straightforward: after Colbourn's work \cite{Colbourn1984} on the complexity of Latin Square completion established the NP-completeness of the problem class, he followed up by motivating the need for efficient heuristics and parallel approaches. Haraguchi and Ono \cite{Haraguchi2014} further analyzed the approximability of Latin Square completion puzzles, providing insights into the theoretical limits of polynomial-time algorithms.

\subsection{Parallel Backtracking and Search}
\label{subsec:backtrack}
Parallel constraint satisfaction has been extensively studied, with various approaches proposed for distributing the search space. The book by Pacheco \cite{Pacheco2011} provides comprehensive coverage of parallel programming patterns, including the master-worker paradigm we employ in the MPI solution, and of course also in the hybrid approach. However, most existing work uses \textit{static work distribution}, whereas our \textit{dynamic multi-level approach} better handles the irregular search space of Futoshiki.
Let us take some moments to assess this statement.
In Sudoku the placement of the constraints is fixed, with a repeating pattern; boxes are identical in shape and size and the \textit{latin square} rule leads to an \textit{all different} constraint. It is easy to see how having fixed rules and symmetric properties leads to well defined search space reduction.

On the other side, Futoshiki introduces the aforementioned inequalities constraints. Due to the fact that they are placed \textit{between} cells, and due to the fact that their position is \textit{arbitrary}, this leads to having a search space reduction which is less efficient, as several assumptions made in Sudoku do not hold.

Research on parallel backtracking typically focuses on either shared-memory or distributed-memory systems in isolation. Among our contributions, we have developed a method which combines both paradigms and demonstrates that hybrid approaches can outperform single-paradigm solutions for irregular workloads.

\subsection{MPI and OpenMP for Irregular Problems}
\label{subsec:mpi_omp_irregular_problems}
\wen{Do we keep this? Could be super nice, but let's double check the references!}
The MPI standard \cite{MPIForum2021} and OpenMP specification \cite{OpenMP2020} provide the foundation for our parallel implementations. Gropp et al. \cite{Gropp1999} discuss various MPI programming patterns, including the master-worker model we adapt for dynamic load balancing.

Rabenseifner et al. \cite{Rabenseifner2009} specifically address hybrid MPI/OpenMP programming, analyzing the benefits and challenges of combining both paradigms. Their work on multi-core clusters directly influenced our hybrid implementation design. We extend their findings by demonstrating that careful management of task \textit{generation factors} to decide whether to rely more on MPI or open mp is crucial for optimal performance.

\subsection{Puzzle Solving Algorithms}
\label{subsec:puzzle_solving_algorithms}
Beyond Sudoku, various researchers have tackled different puzzle types with parallel algorithms, and the main outcome of these algorithms proposed is that the general approach of combining constraint propagation with parallel search is effective across multiple domains. However, most work focuses on puzzles with uniform constraints (like Sudoku's boxes), whereas Futoshiki's arbitrary inequalities presented in \Cref{futoshiki_inequalities} present unique challenges.

Our work differs from existing puzzle solvers in several key aspects:
\begin{itemize}
    \item \textbf{Multi-paradigm approach:} We systematically compare three parallelization strategies.
    \item \textbf{Dynamic depth calculation:} Our work generation adapts to available parallelism to define up to which color a pre-solving strategy can be applied.
    \item \textbf{Configurable granularity:} The \textit{task factor} mechanism in assessing the ratio among available work units and sub problems which are generated allow performance tuning without code changes.
    \item \textbf{Comprehensive evaluation:} We provide detailed analysis of both strong and weak scaling.
\end{itemize}

\subsection{Load Balancing in Parallel Computing}
\label{subsec:load_balancing_in_parallel_computing}
Load balancing for irregular problems remains an active research area. Static approaches suffer from load imbalance when work units have varying difficulty, while dynamic approaches usually face communication overhead. Our solution strikes a balance by generating sufficient work units upfront (based on the calculated depth) while using on-demand distribution to handle variability.
