\section{Future Work}
\label{sec:future_work}
During the course of this report we have presented how our framework compares w.r.t. a standard sequential solution when it comes to the instance of Futoshiki solving. 

There are still some areas to explore in order to shed some light into. Here are presented in a bullet point fashion:

\subsection{Algorithmic Enhancements}
\begin{itemize}
    \item \textbf{Variable Ordering Heuristics:} Implement most-constrained-variable (MCV) and minimum-remaining-values (MRV) heuristics to select cells more intelligently during backtracking.
    
    \item \textbf{Arc Consistency:} Extend constraint propagation to AC-3 or higher levels of consistency for more aggressive pruning.
    \sd{dunno what this is}
    
    \item \textbf{Optimizing search space reduction:} We have seen how the precoloring algorithm already does some of the heavy lifting when it comes to reducing the search space, but there are still some techniques which can be applied. The first one would be to  Implement \textit{clause learning} to avoid redundant exploration of similar subtrees, which can happen quite frequently considering the class of search space. The latter technique is the \textit{Symmetry Breaking}, which conceptually is similar to the Symmetry Breaking (as of we aim to remove some common pattern
    
    \item \textbf{Symmetry Breaking:} Detect and eliminate symmetric solutions to reduce the search space further.
\end{itemize}

\subsection{Parallel Computing Extensions}
\begin{itemize}
    \item \textbf{GPU Acceleration:} Develop CUDA implementation for massively parallel constraint checking and propagation. GPUs could accelerate the pre-coloring phase and enable parallel exploration of multiple search paths.
    
    \item \textbf{Work Stealing:} Replace static work distribution with dynamic work stealing to better handle irregular workloads and heterogeneous systems.
    
    \item \textbf{Asynchronous Communication:} Implement non-blocking MPI communication to overlap computation with data transfer.
    
    \item \textbf{Hierarchical Parallelism:} Explore three-level parallelism: MPI across nodes, OpenMP across cores, and SIMD vectorization within cores.
\end{itemize}

\subsection{Scalability Improvements}
\begin{itemize}
    \item \textbf{Distributed Pre-coloring:} Parallelize the constraint propagation phase, which currently runs redundantly on all processes.
    
    \item \textbf{Adaptive Granularity:} Dynamically adjust work unit size based on runtime performance metrics.
    
    \item \textbf{Fault Tolerance:} Implement checkpointing and recovery mechanisms for long-running computations on large clusters.
    
    \item \textbf{Cloud Deployment:} Adapt the solver for cloud environments with auto-scaling and spot instance support.
\end{itemize}

\subsection{Application to Other Problems}
\begin{itemize}
    \item \textbf{Sudoku Variants:} Extend to Sudoku, Killer Sudoku, and other Latin Square completion puzzles.
    
    \item \textbf{Graph Coloring:} Generalize the framework for arbitrary graph coloring problems.
    
    \item \textbf{Scheduling Problems:} Apply to real-world scheduling with additional constraints (time windows, precedence, resources).
    
    \item \textbf{SAT Solving:} Adapt the parallel framework for Boolean satisfiability problems.
\end{itemize}

\subsection{Machine Learning Integration}
\begin{itemize}
    \item \textbf{Learned Heuristics:} Train neural networks to predict promising search directions based on puzzle features.
    
    \item \textbf{Reinforcement Learning:} Use RL to optimize work distribution and load balancing strategies.
    
    \item \textbf{Pattern Recognition:} Identify common substructures that can be solved independently and cached.
\end{itemize}

\subsection{Software Engineering}
\begin{itemize}
    \item \textbf{Library Development:} Package the framework as a reusable library for constraint satisfaction problems.
    
    \item \textbf{Automatic Parallelization:} Develop compiler techniques to automatically parallelize constraint solvers.
    
    \item \textbf{Performance Modeling:} Create analytical models to predict performance on different architectures.
    
    \item \textbf{Visualization Tools:} Build interactive tools to visualize the search process and parallel execution.
\end{itemize}

These extensions would further improve performance, broaden applicability, and make the framework more accessible to researchers and practitioners working on constraint satisfaction problems.