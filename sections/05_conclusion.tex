\section{Conclusion}
In this paper, we successfully designed and implemented a high-performance solver for the Futoshiki puzzle. Our approach demonstrates the power of combining a sophisticated sequential algorithm with a massively parallel computing framework.

Our findings clearly indicate that the pre-coloring optimization, inspired by the list coloring algorithm from \cite{Sen2024Futoshiki}, is critical for efficient solving. By propagating constraints and reducing the search space upfront, we achieved more than an order of magnitude speedup over a basic backtracking approach.

The MPI-based parallel implementation further extended the solver's capability, enabling it to tackle larger and more difficult puzzles. Our master-worker design proved effective, yielding significant speedups. The performance analysis shows near-ideal scalability for a small number of processes, with an expected and graceful degradation in efficiency as communication overhead becomes a limiting factor at higher process counts.

Ultimately, this project provides a robust template for solving other NP-Complete and combinatorial search problems: first, optimize the sequential algorithm with domain-specific knowledge (like constraint propagation), and second, parallelize the refined algorithm to leverage the power of high-performance computing.