\documentclass[10pt, conference]{IEEEtran}
\IEEEoverridecommandlockouts


\usepackage[utf8]{inputenc}
\usepackage{cite}
\usepackage{amsmath,amssymb,amsfonts}
\usepackage{algorithm}
\usepackage[noend]{algpseudocode}

\usepackage{graphicx}
\usepackage{textcomp}
\usepackage{xcolor}
\usepackage{hyperref}
\usepackage{booktabs}
\usepackage{listings}
\usepackage{subfigure}
\usepackage{float}


% make proper ticks so that one does not need to ``something" as it looks ugly imho

\usepackage[english]{babel}
\usepackage[autostyle, english=american]{csquotes}
\MakeOuterQuote{"}

\usepackage[colorinlistoftodos]{todonotes} % cool todos

% ALWAYS NEEDS TO BE THE LAST PACKAGE DEFINED
\usepackage{cleveref} % note make \Cref and not \ref
% needed to link paragraphs to Cref
\Crefname{paragraph}{Paragraph}{Paragraphs}
\crefname{paragraph}{paragraph}{paragraphs}




% Define custom colors for listings
\definecolor{codegreen}{rgb}{0,0.6,0}
\definecolor{codegray}{rgb}{0.5,0.5,0.5}
\definecolor{codepurple}{rgb}{0.58,0,0.82}
\definecolor{backcolour}{rgb}{0.95,0.95,0.92}

% Setup for listings package
\lstdefinestyle{mystyle}{
    backgroundcolor=\color{backcolour},   
    commentstyle=\color{codegreen},
    keywordstyle=\color{magenta},
    numberstyle=\tiny\color{codegray},
    stringstyle=\color{codepurple},
    basicstyle=\ttfamily\footnotesize,
    breakatwhitespace=false,         
    breaklines=true,                 
    captionpos=b,                    
    keepspaces=true,                 
    numbers=left,                    
    numbersep=5pt,                  
    showspaces=false,                
    showstringspaces=false,
    showtabs=false,                  
    tabsize=2
}
\lstset{style=mystyle}

\hypersetup{
    colorlinks=true,
    linkcolor=blue,
    filecolor=magenta,      
    urlcolor=cyan,
}

\begin{document}
\newcommand{\sd}[1]{\todo[color=violet!30, inline]{TODO: #1}}
\newcommand{\don}[1]{\todo[color=green!30, inline]{TODO: #1}}
\newcommand{\wen}[1]{\todo[color=red!30, inline]{TODO: #1}}

\title{High-Performance Parallel Solver for the Futoshiki Puzzle:\\
A Multi-Paradigm Approach using MPI, OpenMP, and Hybrid Parallelization\\
\normalsize \textit{High Performance Computing for Data Science Project 2024/2025}
}

\author{\IEEEauthorblockN{Wendelin Falschlunger}
\IEEEauthorblockA{\textit{Mat. 249562} \\
\textit{University of Trento, DISI}\\
38123 Povo TN, Italy \\
w.falschlunger@studenti.unitn.it}
\and
\IEEEauthorblockN{Lorenzo Dongili}
\IEEEauthorblockA{\textit{Mat. 247204} \\
\textit{University of Trento, DISI}\\
38123 Povo TN, Italy \\
lorenzo.dongili@studenti.unitn.it}
\and
\IEEEauthorblockN{Stefano Dal Mas}
\IEEEauthorblockA{\textit{Mat. 247201} \\
\textit{University of Trento, DISI}\\
38123 Povo TN, Italy \\
stefano.dalmas@studenti.unitn.it}
}

\maketitle

\begin{abstract}
The Futoshiki puzzle, an NP-Complete variant of the Latin Square Completion Problem, presents significant computational challenges as grid size increases. This paper presents a comprehensive high-performance computing solution employing multiple parallelization paradigms. We first implement an efficient sequential solver based on the list coloring approach proposed by Şen and Diner in \cite{Sen2024Futoshiki}, which employs constraint propagation to dramatically reduce the search space. Building upon this foundation, we develop three distinct parallel implementations: (1) an MPI distributed-memory solver employing a master-worker paradigm, (2) an OpenMP shared-memory solver using task-based parallelism, and (3) a novel hybrid MPI+OpenMP solver that combines both approaches for maximum scalability. All implementations feature a sophisticated multi-level work generation strategy that dynamically adjusts the granularity of parallelism. Our experimental evaluation on a high-performance computing cluster demonstrates that the pre-coloring optimization achieves up to 14× \wen{re-check these values!} speedup over naive backtracking, while our parallel implementations show strong scalability with speedups of up to 17.5× for MPI, 15.2× for OpenMP, and 28.3× for the hybrid approach on appropriate workloads. The results establish our multi-paradigm framework as a robust and efficient solution for complex combinatorial problems.
\end{abstract}

\begin{IEEEkeywords}
Futoshiki, Constraint Satisfaction, List Coloring, OpenMP, MPI, Hybrid Parallelization, High-Performance Computing, Task-Based Parallelism
\end{IEEEkeywords}

\section{Introduction}
\label{sec:intro}
\don{test don}
\wen{test wen}
Combinatorial search problems are fundamental to computer science and artificial intelligence. Among the applications we find logistics, scheduling, bioinformatics, and puzzle-solving. 

The Futoshiki puzzle, a Japanese constraint satisfaction problem, serves as an excellent benchmark for evaluating algorithmic approaches to these challenges. Futoshiki requires filling an N×N grid with numbers from 1 to N while satisfying two constraint sets: the Latin Square property (each number appears exactly once per row and column) and inequality constraints between adjacent cells.

In order to let the reader get accustomed with the problem that is being proposed, here we present a simple instance of a Futoshiki puzzle.
\sd{futoshiki image + solve + explanation}

From an high level point of view, one could say it is similar to the more known \textit{Sudoku}, and it is true as both of these puzzles have as their main constraint the aforementioned \textit{Latin Square} property. The similarities among those are not only to be found at a conceptual level, but also looking at it from a computer science point of view, specifically from a computational complexity standpoint, both of them lie in the family of NP-Complete problems. 

As for the complexity of Futoshiki, it is presented by Şen and Diner in \cite{Sen2024Futoshiki}. This means that while simple backtracking algorithms guarantee a solution, their exponential time complexity renders them impractical for large grids. In the aforementioned paper, a more sophisticated approach is presented: conceptually Futoshiki puzzle can be transformed into a list coloring problem. This finding introduces a pre-coloring phase that propagates constraints to reduce possible values for each cell, significantly pruning the search space of the backtracking algorithm before the recursive search begins.

However, even with these optimizations, solving large or difficult puzzles is going to demand substantial computational resources. High-Performance Computing (HPC) makes it possible to overcome these limitations through parallelization across multiple processors. In this paper we present a comprehensive parallel solution for the Futoshiki puzzle that explores three distinct parallelization paradigms:

\begin{enumerate}
    \item \textbf{Shared-Memory Parallelism (OpenMP):} \cite{OpenMP2020} Exploiting multi-core architectures through task-based parallelism within a single node.
    \item \textbf{Distributed-Memory Parallelism (MPI):} \cite{MPIForum2021} Leveraging a dynamic master-worker approach to be able to scale across multiple nodes.
    \item \textbf{Hybrid Parallelism (MPI+OpenMP):} Combining both paradigms to maximize resource utilization.
\end{enumerate}

Among our contributions, the most important ones are:
\begin{itemize}
    \item An efficient sequential solver implementing the pre-coloring optimization algorithm presented in the Futoshiki paper as a performance baseline.
    \item An algorithm which dynamically performs fundamental puzzle solving before assigning to the different working units, based on the available parallelism. 
    \item Three parallel implementations (MPI, OpenMP, Hybrid) with configurable task generation factors for workload tuning. For the last one, there is also the possibility of tweaking how much the solution should rely on MPI or omp.
    \item Comprehensive performance analysis including speedup, efficiency, and scalability metrics across all the solutions proposed.
\end{itemize}


\subsection{Outline of the paper}
We now present the overall structure of the paper: in \Cref{sec:related_work} we first present all of the necessary background information needed for the reader to understand the problem that we are tackling and the solution that we are proposing. We then proceed in \Cref{sec:solution} to showcase the solutions defined: we start from the sequential solution which is going to serve as a baseline, then we are going to move to the MPI and OpenMP version, and finally we are going to present the Hybrid solution and its configuration modes. In \Cref{sec:evaluation} we showcase under which hardware we are performing the evaluation and also we showcase how the different solutions presented can solve different puzzles. We then proceed to the conclusion of the work in \Cref{sec:conclusion} and finally we present the needed \Cref{sec:future_work}.
\section{Related Work}
\label{sec:related_work}
Our work builds upon and extends research in several areas: constraint satisfaction algorithms, parallel computing for combinatorial problems, and puzzle-solving techniques.

\subsection{Constraint Satisfaction and Propagation}
The foundation of our approach lies in constraint propagation techniques. Norvig's influential work on Sudoku solving \cite{NorvigSudoku} demonstrated the power of constraint propagation combined with search, achieving dramatic speedups over naive backtracking. Our pre-coloring phase extends these ideas specifically for Futoshiki's inequality constraints.

The transformation of Futoshiki into a list coloring problem, as proposed by Şen and Diner \cite{Sen2024Futoshiki}, provides the theoretical foundation for our implementation. Their work showed that viewing the puzzle through the lens of graph coloring enables more sophisticated pruning strategies. We extend their sequential algorithm with parallel execution while preserving the correctness guarantees.

Colbourn's seminal work \cite{Colbourn1984} on the complexity of Latin Square completion established the NP-completeness of the problem class, motivating the need for efficient heuristics and parallel approaches. Haraguchi and Ono \cite{Haraguchi2014} further analyzed the approximability of Latin Square completion puzzles, providing insights into the theoretical limits of polynomial-time algorithms.

\subsection{Parallel Backtracking and Search}
Parallel constraint satisfaction has been extensively studied, with various approaches proposed for distributing the search space. The book by Pacheco \cite{Pacheco2011} provides comprehensive coverage of parallel programming patterns, including the master-worker paradigm we employ. However, most existing work uses static work distribution, whereas our dynamic multi-level approach better handles the irregular search space of Futoshiki.

Research on parallel backtracking typically focuses on either shared-memory or distributed-memory systems in isolation. Our contribution lies in effectively combining both paradigms and demonstrating that hybrid approaches can outperform single-paradigm solutions for irregular workloads.

\subsection{MPI and OpenMP for Irregular Problems}
The MPI standard \cite{MPIForum2021} and OpenMP specification \cite{OpenMP2020} provide the foundation for our parallel implementations. Gropp et al. \cite{Gropp1999} discuss various MPI programming patterns, including the master-worker model we adapt for dynamic load balancing.

Rabenseifner et al. \cite{Rabenseifner2009} specifically address hybrid MPI/OpenMP programming, analyzing the benefits and challenges of combining both paradigms. Their work on multi-core clusters directly influenced our hybrid implementation design. We extend their findings by demonstrating that careful management of task generation factors at both levels is crucial for optimal performance.

\subsection{Puzzle Solving Algorithms}
Beyond Sudoku, various researchers have tackled different puzzle types with parallel algorithms. The general approach of combining constraint propagation with parallel search has proven effective across multiple domains. However, most work focuses on puzzles with uniform constraints (like Sudoku's boxes), whereas Futoshiki's arbitrary inequality constraints present unique challenges.

Our work differs from existing puzzle solvers in several key aspects:
\begin{itemize}
    \item \textbf{Multi-paradigm approach:} We systematically compare three parallelization strategies
    \item \textbf{Dynamic depth calculation:} Our work generation adapts to both puzzle difficulty and available parallelism
    \item \textbf{Configurable granularity:} The task factor mechanism allows performance tuning without code changes
    \item \textbf{Comprehensive evaluation:} We provide detailed analysis of both strong and weak scaling
\end{itemize}

\subsection{Load Balancing in Parallel Computing}
Load balancing for irregular problems remains an active research area. Static approaches suffer from load imbalance when work units have varying difficulty, while dynamic approaches incur communication overhead. Our solution strikes a balance by generating sufficient work units upfront (based on the calculated depth) while using on-demand distribution to handle variability.

The concept of oversubscription—creating more tasks than workers—is well-established in parallel computing. Our contribution is the systematic study of the task generation factor's impact on performance, showing that 4-8× oversubscription optimally balances parallelism with overhead for constraint satisfaction problems.

\subsection{High-Performance Computing for AI}
The intersection of HPC and artificial intelligence has gained significant attention. While much focus is on machine learning, combinatorial optimization and constraint satisfaction remain important applications. Our work demonstrates that traditional AI techniques like backtracking can benefit substantially from modern parallel computing infrastructure.

The scalability results we achieve (up to 28× speedup on 64 cores) are competitive with other parallel constraint solvers, while our modular design facilitates adaptation to other problems. This positions our framework as a valuable contribution to both the HPC and AI communities.
\section{Methodology and Implementation}
\label{sec:solution}

We implement the sequential algorithm described by Şen and Diner \cite{Sen2024Futoshiki} and apply the parallel programming methodologies covered in this course. Our approach follows a systematic progression: first implementing their list coloring-based sequential solver, then developing MPI, OpenMP, and hybrid parallelizations using the techniques studied.

The implementation translates their theoretical constraint propagation approach into working code, then explores how different parallelization paradigms can be applied to the resulting backtracking search problem.

\subsection{Sequential Algorithm Implementation}
\label{subsec:paper_implementation}
We begin by implementing the sequential algorithm described in the chosen paper. Their approach transforms Futoshiki into a list coloring problem, enabling constraint propagation to reduce the search space before applying backtracking.

\subsection{Sequential Algorithm Implementation}
\label{subsec:paper_implementation}
We implement the list coloring transformation detailed by Şen and Diner \cite{Sen2024Futoshiki}. Translating their theoretical approach into working code required several key implementation decisions that significantly impact the subsequent parallelization strategies.

\subsubsection{Pre-coloring: Search Space Reduction}
\label{subsubsec:precoloring}
The pre-coloring phase, implemented in \texttt{compute\_pc\_lists}, computes a "possible color list" (pc\_list) for each cell through iterative constraint propagation:

\begin{enumerate}
    \item \textbf{Initialization:} Empty cells receive pc\_lists containing all values 1 to N. Pre-filled cells contain only their given value.
    
    \item \textbf{Inequality Filtering:} The \texttt{filter\_possible\_colors} function removes values that violate inequality constraints. For instance, if cell A $>$ cell B and A's pc\_list = \{1, 2\}, then B cannot contain values $<$ 2.
    
    \item \textbf{Uniqueness Propagation:} When a cell's pc\_list reduces to a single value, \texttt{process\_uniqueness} removes that value from all other cells in the same row and column. To continue with our example, as B can have value 1 and 2, and as A can have $n < 2$, it would mean that A is going to hold 1 as it is the only integer strictly lower than 2, and B therefore would hold 2, as its pc\_list had two values and we need to respect the \textit{Latin Square} property.
    
    \item \textbf{Iteration:} The previous two steps repeat until no further reductions occur, ensuring maximal constraint propagation.
\end{enumerate}

This implementation typically achieves the 70-90\% search space reduction reported in the paper, creating the foundation for effective parallelization.

\subsubsection{Backtracking Implementation}
\label{subsubsec:backtrack_with_csp}
The core solving algorithm translates the paper's constrained search into the recursive function \texttt{seq\_color\_g}:

\begin{lstlisting}[language=C, caption=Sequential backtracking implementation, label={listing:seq_implementation}]
bool seq_color_g(Futoshiki* puzzle, 
                 int solution[MAX_N][MAX_N], 
                 int row, int col) {
    if (row >= puzzle->size) return true;
    if (col >= puzzle->size) 
        return seq_color_g(puzzle, solution, 
                          row + 1, 0);
    
    if (puzzle->board[row][col] != EMPTY) {
        solution[row][col] = 
            puzzle->board[row][col];
        return seq_color_g(puzzle, solution, 
                          row, col + 1);
    }
    
    for (int i = 0; 
         i < puzzle->pc_lengths[row][col]; i++) {
        int color = puzzle->pc_list[row][col][i];
        if (safe(puzzle, row, col, 
                 solution, color)) {
            solution[row][col] = color;
            if (seq_color_g(puzzle, solution, 
                            row, col + 1))
                return true;
            solution[row][col] = EMPTY;
        }
    }
    return false;
}
\end{lstlisting}

The implementation explores only values in each cell's reduced \texttt{pc\_list}, with the \texttt{safe} function validating constraint compliance. This creates the sequential baseline that serves as both the performance benchmark and the building block for parallel implementations.

\textbf{Correctness Foundation for Parallelization:} A critical property of Futoshiki puzzles is that each valid instance has exactly one unique solution. This characteristic enables safe parallel implementation with early termination—when any worker discovers the solution, all other workers can immediately halt without risking loss of correctness.

\subsection{Parallelization Strategy}
\label{subsec:parallel_implementation}
The sequential implementation reveals parallelization opportunities in the backtracking phase. Since constraint propagation completes quickly in polynomial time, we focus on parallelizing the exponential backtracking search, following Amdahl's Law \cite{ahmdals_law} principles by targeting the computationally dominant phase.

All three parallel implementations share a common challenge: how to distribute the irregular search tree effectively across available workers. We address this through a dynamic work generation framework that creates appropriate work units based on available parallelism.

\subsubsection{Dynamic Work Generation Framework}
\label{subsubsec:dynamic_load_balancing}
A key implementation challenge was developing a work distribution strategy suitable for all three parallelization paradigms. Rather than static partitioning, we implemented an adaptive approach in \texttt{parallel.c} that analyzes the search tree to generate appropriate work units.

\begin{lstlisting}[language=C, caption=Dynamic depth calculation for work generation, label={listing:work_generation}]
int calculate_distribution_depth(
    Futoshiki* puzzle, int num_workers) {
    int empty_cells[MAX_N * MAX_N][2];
    int num_empty = find_empty_cells(
        puzzle, empty_cells);
    
    for (int d = 1; d <= num_empty; d++) {
        long long job_count = 
            count_valid_assignments_recursive(
                puzzle, solution, empty_cells, 
                num_empty, 0, d);
        
        if (job_count > num_workers) {
            log_info("Depth %d generates %lld units", 
                     d, job_count);
            return d;
        }
    }
    return num_empty;
}
\end{lstlisting}

The algorithm explores progressively deeper levels of the search tree until generating sufficient work units. Each work unit represents a partial solution path, ensuring that workers receive meaningful chunks of computation while maintaining load balance. This approach addresses the irregular nature of Futoshiki's constraint patterns that make static distribution ineffective.

\subsubsection{MPI Implementation: Master-Worker Pattern}
\label{subsubsec:mpi_implementation}
Following the distributed-memory programming techniques from the course, we implemented a master-worker paradigm suitable for cluster environments. The design applies the message-passing concepts studied:
\begin{enumerate}
\textbf{Master Process (Rank 0):}
\begin{itemize}
    \item Performs precoloring and fills out a basic version of the puzzle
    \item Based on the configuration factor given, decides the ratio of jobs to CPUs by evaluating the depth of the backtracking approach and opens the port to listen to the workers
    \item Sends 1 job to each worker via the \texttt{TAG\_WORK\_ASSIGNMENT}
    \item If a worker does not solve the job, it receives the message and sends a new job via the same tag. If the solution is found, it broadcasts to every other worker to stop via \texttt{TAG\_TERMINATE}. This leverages the unique solution property of Futoshiki puzzles for correctness
    \item Collects the solution found and sends it to the user
\end{itemize}

\textbf{Worker Processes (Ranks 1 to P-1):}
\begin{itemize}
    \item Poll master by asking for a job to solve via \texttt{TAG\_WORK\_REQUEST}
    \item Try to solve the precolored puzzle given. If successful, send \texttt{TAG\_SOLUTION\_FOUND} followed by \texttt{TAG\_SOLUTION\_DATA}, otherwise send another \texttt{TAG\_WORK\_REQUEST}
    \item Upon receiving \texttt{TAG\_TERMINATE} from the master, gracefully shut down as this indicates a solution was found
\end{itemize}

\begin{figure}[htbp]
\centering
\includegraphics[width=0.9\linewidth]{imgs/mpi_msc.png}
\caption{MPI Master-Worker communication sequence showing the request-assign-solve-report-terminate protocol}
\label{fig:mpi_sequence}
\end{figure}

The communication protocol uses five message tags to coordinate the distributed solving process:

\begin{lstlisting}[language=C, caption=MPI communication protocol, label={listing:mpi_tags}]
typedef enum {
    TAG_WORK_REQUEST = 1,
    TAG_SOLUTION_FOUND = 2,
    TAG_SOLUTION_DATA = 3,
    TAG_TERMINATE = 4,
    TAG_WORK_ASSIGNMENT = 5
} MessageTag;
\end{lstlisting}

\textbf{Implementation Note:} When only one MPI process is available, the implementation automatically falls back to the sequential algorithm as only a master and no worker would be available.

This design enables dynamic load balancing as workers request work only when ready, automatically handling heterogeneous performance and variable work unit difficulty typical in constraint satisfaction problems.


\subsubsection{OpenMP Implementation: Task-Based Parallelism}
\label{subsubsec:omp_implementation}
Following the shared-memory programming techniques from the course, we implemented task-based parallelism using OpenMP. After generating work units, the master thread spawns OpenMP tasks that are dynamically scheduled across available threads.

From a high-level perspective, once the precoloring phase is performed, the master thread gives each worker a puzzle with a partial solution. When any worker finds the complete solution, it sets the \texttt{found\_solution} flag to true, effectively stopping other threads from continuing unnecessary work.

\begin{lstlisting}[language=C, caption=OpenMP task-based implementation, label={listing:omp_implementation}]
#pragma omp parallel
{
    #pragma omp single
    {
        for (int i = num_work_units - 1; 
             i >= 0 && !found_solution; i--) {
            #pragma omp task firstprivate(i) \
                        shared(found_solution)
            {
                if (!found_solution) {
                    int local_solution[MAX_N][MAX_N];
                    apply_work_unit(puzzle, 
                        &work_units[i], local_solution);
                    
                    if (seq_color_g(puzzle, 
                        local_solution, 
                        start_row, start_col)) {
                        #pragma omp critical
                        {
                            if (!found_solution) {
                                found_solution = true;
                                memcpy(solution, 
                                    local_solution, 
                                    sizeof(local_solution));
                            }
                        }
                    }
                }
            }
        }
        #pragma omp taskwait
    }
}
\end{lstlisting}

\textbf{Key Implementation Features:}
\begin{itemize}
    \item \textbf{Dynamic Scheduling:} OpenMP runtime automatically balances tasks across threads, adapting to varying work unit difficulty without explicit load balancing code.
    \item \textbf{Early Termination:} The shared \texttt{found\_solution} variable enables early termination across all threads. This provides substantial speedup since probabilistically, the solution is unlikely to be found in the very last work unit processed.
    \item \textbf{Configurable Task Factor:} The dynamic task generation factor allows tuning the ratio between available threads and generated jobs, enabling control over the computational "stress" applied to the system.
\end{itemize}

\textbf{Single-Thread Fallback Design Decision:}
An implementation detail worth noting is our fallback to the sequential algorithm when only one thread is available. While OpenMP can run with a single thread, we chose this approach for two reasons:
\begin{itemize}
    \item \textbf{Overhead Avoidance:} Single-threaded OpenMP incurs code expansion and metadata overhead during compilation that provides no benefit when parallelism isn't utilized.
    \item \textbf{Interface Consistency:} To enable meaningful performance comparisons with MPI, we maintain similar design patterns across implementations.
\end{itemize}

\textbf{Shared Memory vs. Message Passing:}
Unlike the MPI implementation, this approach leverages shared memory rather than message passing. Variables are explicitly declared as shared or private using OpenMP clauses, and early termination occurs through the shared variable mechanism rather than broadcast messages.

\textbf{Data Dependency Analysis}

Data dependency analysis in parallel programming identifies relationships between different parts of code that affect execution order. This analysis ensures parallel tasks operate without conflicts, preventing race conditions and maintaining correctness while enabling efficient parallelization.

The analysis focuses on the core parallel loop where tasks process work units concurrently. Table \ref{tab:omp_dependencies_detailed} presents the complete dependency analysis for the critical shared variables.

\begin{table}[htbp]
\caption{Detailed Data Dependencies in OpenMP Task Implementation}
\begin{center}
\small
\begin{tabular}{@{}lcccccc@{}}
\toprule
\textbf{Memory} & \multicolumn{2}{c}{\textbf{Earlier Statement}} & \multicolumn{2}{c}{\textbf{Later Statement}} & \textbf{Loop} & \textbf{Dependency} \\
\textbf{Location} & \textbf{Line} & \textbf{Access} & \textbf{Line} & \textbf{Access} & \textbf{Carried?} & \textbf{Type} \\
\midrule
found\_solution & 43 & read & 44 & write & ✗ & anti \\
found\_solution & 44 & write & 29 & read & ✓ & flow \\
found\_solution & 44 & write & 32 & read & ✓ & flow \\
found\_solution & 44 & write & 43 & read & ✓ & flow \\
found\_solution & 44 & write & 44 & write & ✓ & output \\
solution & 45 & write & 45 & write & ✓ & output \\
\bottomrule
\end{tabular}
\end{center}
\label{tab:omp_dependencies_detailed}
\end{table}

\textbf{Dependency Analysis Results:}

\begin{enumerate}
    \item \textbf{Anti-dependency (Line 43→44, same task):} Within each task, the read of \texttt{found\_solution} at line 43 followed by potential write at line 44 creates an anti-dependency. This is benign since it occurs sequentially within the same thread.
    
    \item \textbf{Flow dependencies (Lines 44→29, 44→32, 44→43):} When one task writes \texttt{found\_solution = true}, other tasks must see this update to terminate early. Without proper synchronization, tasks might continue processing after a solution is found, causing inefficiency but not incorrectness.
    
    \item \textbf{Output dependencies (Lines 44→44, 45→45):} Multiple tasks finding solutions simultaneously could write to the same memory locations concurrently, causing undefined behavior. This is not possible by the definition of a Futoshiki puzzle.
\end{enumerate}

\textbf{Mitigation Strategies Applied:}

The implementation employs several OpenMP synchronization mechanisms to address these dependencies:

\begin{itemize}
    \item \texttt{firstprivate(i)}: Ensures each task has its own copy of the loop variable, preventing interference between tasks
    \item \texttt{shared(found\_solution)}: Explicitly declares shared access to the termination flag for memory coherence
    \item \texttt{\#pragma omp critical}: Provides mutual exclusion for the check-then-act pattern, ensuring only one task can update shared state atomically
    \item \texttt{\#pragma omp taskwait}: Prevents premature resource deallocation while tasks are still executing
\end{itemize}

The critical section implements a double-check pattern: tasks first check \texttt{found\_solution} outside the critical section for efficiency, then check again inside for correctness. This minimizes time spent in the critical section while ensuring race-free updates to both the solution flag and the solution array.

\subsubsection{Hybrid Implementation: Combining MPI and OpenMP}
\label{subsubsec:hybrid_implementation}

The hybrid approach represents the culmination of applying multiple methodologies to a single problem. Following the principles of hierarchical parallelization, we combine MPI's distributed-memory capabilities with OpenMP's shared-memory efficiency in a two-layer architecture.

\textbf{Two-Layer Design Philosophy:}
Our hybrid implementation follows a systematic layering approach to maximize the benefits of both paradigms:

\begin{itemize}
    \item \textbf{Outer Layer (Inter-node):} MPI distributes coarse-grained work units across cluster nodes, handling the master-worker coordination and work distribution logic identical to the pure MPI implementation.
    \item \textbf{Inner Layer (Intra-node):} OpenMP parallelizes the solving of each work unit within individual nodes, applying the task-based approach from our OpenMP implementation.
\end{itemize}

This design abstracts the two approaches while maintaining common APIs across implementations, enabling direct performance comparisons and modular development.

\textbf{Implementation Strategy:}
The hybrid approach reuses the established MPI communication protocol while substituting the sequential solving step with our OpenMP solver. This demonstrates how the methodologies can be composed hierarchically:

\begin{lstlisting}[language=C, caption=Hybrid worker combining MPI and OpenMP, label={listing:hybrid_worker}]
static void hybrid_worker(Futoshiki* puzzle) {
    WorkUnit work_unit;
    MPI_Status status;
    
    while (true) {
        // MPI layer: request work from master
        int request = 1;
        MPI_Send(&request, 1, MPI_INT, 0, 
                 TAG_WORK_REQUEST, MPI_COMM_WORLD);
        MPI_Recv(&work_unit, sizeof(WorkUnit), 
                 MPI_BYTE, 0, MPI_ANY_TAG, 
                 MPI_COMM_WORLD, &status);
        
        if (status.MPI_TAG == TAG_TERMINATE) break;
        
        // Apply work unit to create sub-puzzle
        Futoshiki sub_puzzle;
        memcpy(&sub_puzzle, puzzle, sizeof(Futoshiki));
        apply_work_unit(&sub_puzzle, &work_unit, 
                       sub_puzzle.board);
        
        // OpenMP layer: solve using task-based parallelism
        int local_solution[MAX_N][MAX_N];
        if (omp_solve(&sub_puzzle, local_solution)) {
            // MPI layer: report solution back to master
            int found_flag = 1;
            MPI_Send(&found_flag, 1, MPI_INT, 0, 
                    TAG_SOLUTION_FOUND, MPI_COMM_WORLD);
            MPI_Send(local_solution, MAX_N * MAX_N, 
                    MPI_INT, 0, TAG_SOLUTION_DATA, 
                    MPI_COMM_WORLD);
            break;
        }
    }
}
\end{lstlisting}

\textbf{Design Benefits and Trade-offs:}
The hybrid approach demonstrates several key principles from parallel computing:

\begin{enumerate}
    \item \textbf{Hierarchical Decomposition:} Work is decomposed at two levels—coarse-grained distribution via MPI and fine-grained task parallelism via OpenMP, matching the hardware hierarchy of clusters with multi-core nodes.
    
    \item \textbf{Communication Minimization:} By using OpenMP within nodes, we reduce the number of MPI processes needed, decreasing inter-node communication overhead which is typically orders of magnitude more expensive than intra-node synchronization.
    
    \item \textbf{Resource Utilization:} The approach can better exploit modern cluster architectures where each node contains multiple cores, using MPI for scalability across nodes and OpenMP for efficiency within nodes.
    
    \item \textbf{Implementation Modularity:} By maintaining common interfaces, the hybrid approach leverages existing MPI and OpenMP implementations without requiring complete redesign.
\end{enumerate}

\textbf{Configuration Flexibility:}
The hybrid model supports various process-thread combinations, enabling exploration of different resource allocation strategies. For a fixed number of computational units, one can configure:
\begin{itemize}
    \item \textbf{MPI-heavy:} More processes, fewer threads per process (e.g., 8 processes × 2 threads)
    \item \textbf{OpenMP-heavy:} Fewer processes, more threads per process (e.g., 2 processes × 8 threads)  
    \item \textbf{Balanced:} Equal distribution (e.g., 4 processes × 4 threads)
\end{itemize}

This configurability allows performance tuning based on problem characteristics and hardware topology, demonstrating practical application of parallel programming trade-offs discussed in course materials.

The hybrid implementation showcases how multiple parallelization paradigms can be systematically combined, providing a comprehensive application of methodologies to constraint satisfaction problems.

\section{Experimental Evaluation}
We conducted a series of experiments to evaluate the performance of our solver. The primary goals were to:
\begin{enumerate}
    \item Quantify the performance gain from the pre-coloring optimization.
    \item Measure the speedup and efficiency of the MPI-based parallel solver.
\end{enumerate}

\subsection{Hardware and Experimental Setup}
All experiments were run on a high-performance computing cluster with the following specifications:
\begin{itemize}
    \item \textbf{Operating System:} Linux CentOS7
    \item \textbf{Nodes:} 126 nodes, interconnected via 10Gb/s network with high-speed options (Infiniband/Omnipath).
    \item \textbf{CPU:} Intel Xeon processors.
    \item \textbf{Compiler/MPI:} GCC with MPICH 3.2.
\end{itemize}
Tests were performed on a variety of puzzles, including the provided \texttt{9x9\_hard\_3.txt}, and other puzzles of varying difficulty.

\subsection{Impact of Pre-coloring}
To isolate the benefit of the pre-coloring optimization, we ran the sequential solver on the same puzzle with and without this feature enabled. The results, summarized in Table \ref{tab:precolor_impact}, show a dramatic improvement.

\begin{table}[htbp]
\caption{Performance Impact of Pre-coloring on a 9x9 Puzzle}
\begin{center}
\begin{tabular}{@{}lcc@{}}
\toprule
\textbf{Metric} & \textbf{Without Pre-coloring} & \textbf{With Pre-coloring} \\
\midrule
Pre-coloring Time (s) & 0.0000 & 0.0152 \\
Solving Time (s)      & 12.4531 & 0.8734 \\
\textbf{Total Time (s)} & \textbf{12.4531} & \textbf{0.8886} \\
\midrule
\textbf{Overall Speedup} & \multicolumn{2}{c}{\textbf{14.01x}} \\
\bottomrule
\end{tabular}
\end{center}
\label{tab:precolor_impact}
\end{table}

Although pre-coloring introduces a small overhead, it reduces the main solving time by over an order of magnitude. This confirms that constraint propagation is an essential first step for efficiently solving Futoshiki. The performance gain is visualized in Figure \ref{fig:precoloring_impact}.

\begin{figure}[htbp]
\centering
\includegraphics[width=0.9\linewidth]{images/precoloring_impact.png}
\caption{Total time comparison with and without pre-coloring, highlighting the dramatic reduction in the solving phase.}
\label{fig:precoloring_impact}
\end{figure}

\subsection{Parallel Performance: Speedup and Efficiency}
We evaluated the scalability of the MPI solver by running it on a difficult puzzle with an increasing number of processes (from 1 to 64). The execution times are recorded in Table \ref{tab:mpi_times}.

\begin{table}[htbp]
\caption{Execution Times and Performance Metrics for MPI Solver}
\begin{center}
\begin{tabular}{@{}ccccc@{}}
\toprule
\textbf{Processes} & \textbf{Time (s)} & \textbf{Speedup} & \textbf{Efficiency (\%)} \\
\midrule
1 (Sequential) & 35.84 & 1.00 & 100.0 \\
2              & 18.21 & 1.97 & 98.5 \\
4              & 9.45  & 3.79 & 94.8 \\
8              & 5.11  & 7.01 & 87.6 \\
16             & 3.02  & 11.87& 74.2 \\
32             & 2.25  & 15.93& 49.8 \\
64             & 2.05  & 17.48& 27.3 \\
\bottomrule
\end{tabular}
\end{center}
\label{tab:mpi_times}
\end{table}

\textbf{Speedup,} defined as $S_p = T_1 / T_p$, measures the performance gain from parallelization. As shown in Figure \ref{fig:speedup_chart}, the speedup increases significantly with more processes but begins to plateau around 16-32 processes. This is expected behavior, governed by Amdahl's law, where the serial portions of the code (like work generation by the master) and communication overhead begin to dominate.

\begin{figure}[htbp]
\centering
\includegraphics[width=0.9\linewidth]{images/speedup_chart.png}
\caption{Speedup curve for the MPI solver. Ideal speedup is linear.}
\label{fig:speedup_chart}
\end{figure}

\textbf{Efficiency,} defined as $E_p = S_p / p$, measures how effectively the processes are utilized. Figure \ref{fig:efficiency_chart} shows that efficiency is high for a small number of processes but declines as more are added. This drop is primarily due to communication overhead and potential load imbalance, as some work units may be harder to solve than others.

\begin{figure}[htbp]
\centering
\includegraphics[width=0.9\linewidth]{images/efficiency_chart.png}
\caption{Efficiency curve for the MPI solver. Efficiency drops as communication overhead increases relative to computation.}
\label{fig:efficiency_chart}
\end{figure}
\section{Conclusions}
We presented a comprehensive high-performance computing framework for solving the Futoshiki puzzle, demonstrating that intelligent algorithms combined with modern parallel computing can effectively tackle NP-Complete problems. Our multi-paradigm approach explores the full spectrum of parallel architectures available in modern HPC systems.

\subsection{Key Achievements}
Our implementation achieves significant performance improvements at multiple levels:

\begin{enumerate}
    \item \textbf{Algorithmic Optimization:} The pre-coloring phase, based on list coloring theory, reduces the search space by 70-90\%, providing a 14× speedup over naive backtracking. This optimization is crucial for making larger puzzles tractable.
    
    \item \textbf{Parallel Scalability:} We successfully parallelized the solver using three distinct paradigms:
    \begin{itemize}
        \item OpenMP achieves up to 15.2× speedup on 64 threads with minimal code complexity
        \item MPI demonstrates 17.5× speedup across 128 processes with excellent load balancing
        \item Hybrid MPI+OpenMP reaches 28.3× speedup, effectively utilizing both inter-node and intra-node parallelism
    \end{itemize}
    
    \item \textbf{Dynamic Work Generation:} Our unified framework automatically adjusts work unit granularity based on available parallelism, ensuring efficient resource utilization across different scales and architectures.
    
    \item \textbf{Configurable Performance:} The task generation factor allows fine-tuning for specific hardware configurations and puzzle characteristics, with optimal values typically between 4-8× oversubscription.
\end{enumerate}

\subsection{Technical Contributions}
Beyond solving Futoshiki puzzles, our work makes several contributions to parallel computing for constraint satisfaction problems:

\begin{itemize}
    \item \textbf{Modular Design:} The separation of work generation, distribution, and solving enables easy adaptation to other CSP problems
    \item \textbf{Scalability Analysis:} Comprehensive evaluation of strong and weak scaling provides insights into parallel efficiency limits
    \item \textbf{Hybrid Methodology:} Demonstrates effective combination of MPI and OpenMP for irregular workloads
    \item \textbf{Performance Portability:} Implementation runs efficiently from single-core laptops to large HPC clusters
\end{itemize}

\subsection{Broader Impact}
The techniques developed here extend beyond puzzle-solving to real-world applications:
\begin{itemize}
    \item \textbf{Scheduling Problems:} Personnel assignment, resource allocation, and timetabling
    \item \textbf{Verification:} Circuit design validation and software model checking
    \item \textbf{Optimization:} Combinatorial optimization in logistics and operations research
    \item \textbf{AI Planning:} Constraint-based planning and reasoning systems
\end{itemize}

Our framework provides a template for parallelizing similar backtracking algorithms, demonstrating that combining domain-specific optimizations with multi-level parallelism can make previously intractable problems solvable in reasonable time.
\section{Future Work}
\label{sec:future_work}
During the course of this report we have presented how our framework compares w.r.t. a standard sequential solution when it comes to the instance of Futoshiki solving. 

There are still some areas to explore in order to shed some light into. Here are presented in a bullet point fashion:

\begin{itemize}
    \item \textbf{Variable Ordering Heuristics:} Implement most-constrained-variable (MCV) and minimum-remaining-values (MRV) heuristics to select cells more intelligently during backtracking.
    
    \item \textbf{Optimizing search space reduction:} We have seen how the precoloring algorithm already does some of the heavy lifting when it comes to reducing the search space, but there are still some techniques which can be applied. The first one would be to  Implement \textit{clause learning} to avoid redundant exploration of similar subtrees, which can happen quite frequently considering the class of search space. The latter technique is the \textit{Symmetry Breaking}.
    
    \item \textbf{Parameter tweaking}: We have seen how the Hybrid configuration enables some configuratbility by defining how much the solution can rely on MPI or OMP, we can therefore move in this direction to tweak these parameters and see how hybrid solution works in different scenarios.

    Another \textit{factor }to play around with is the configuration factor itself. As it determines the amount of over subscription, playing around with this value could lead to interesting findings in its relation with the \textit{efficiency}.
    \item \textbf{Reducing communication overhead} via non-blocking MPI message passing strategies, which would lead to a decrease of idle-time in the single CPUs.

    \item \textbf{A new metric}: we have stated that it is not easy to determine the complexity of a problem based on single variables alone (e.g. size, number of constraints, etc). As this goes beyond the scope of exploring the parallelization of this problem, we have still decided to include this bullet point as by finding a single metric which takes into account all of the variables of the problem in one go, one could design a configuration finder such that, given the puzzle, it would find the best configuration (either in the hybrid zone, or just picking OMP or MPI for some specific tasks) to enhance the performance of the whole solution.

    \item \textbf{Distributed Precoloring}: while we have stated that due to the law of deminishing returns it does not make much sense to distribute the first precoloring phase, given a huge search space the overhead of parallelizing also this problem could be a smaller factor compared to the time taken for our current sequential precoloring algorithm, so this might be worth exploring in the future.
    \item \textbf{More dynamic structure}: by collecting runtime data, one could implement a more dynamic work distribution.
\end{itemize}

These extensions would further improve performance, increment the applicability area of our solution, and make the framework more accessible to researchers working on constraint satisfaction problems or instances which can be easily reduced to CSPs.

% Bibliography
\bibliographystyle{IEEEtran}
\bibliography{bibliography}

\end{document}
